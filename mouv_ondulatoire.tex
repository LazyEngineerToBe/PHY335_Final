\section{Mouvement Ondulatoire}
\vspace{-2\baselineskip}
\subsection{Onde sinusoïdale}

\begin{tabular}{lll}
Valeur & Formule & Unitée \\\hline
Longueur d'onde & \(\lambda = cT = c/f\) & m \\%[5pt]
Vitesse de propagation & \(c=\sqrt{F/\mu} \) & m/s \\%\hline\rule{0pt}{15pt}\hspace{-6pt} 
Const. de propagation & \(k = \omega/c = 2\pi/\lambda\)& m$^{-1}$\\
Densité lin. & \(\mu = F/c^2\) & kg/m\\
Vitesse max. & \( v_{\textit{max}}=\omega A\) & m/s\\%[5pt]\hline\rule{0pt}{15pt}
\end{tabular}%
% }


\subsection{Équations du mouvement}
\begin{align*}
    y(x,t) &= A\sin (\omega t \pm kx+\phi)\\
    v(x,t) &=\omega A\cos (\omega t \pm kx+\phi)\\
    a(x,t) &= -\omega^2 A\sin (\omega t \pm kx+\phi)
\end{align*}
\subsubsection{Sens de propagation}
\begin{center}
    \begin{tabular}{cl}
        $-$ & vers la droite ($x>0$)\\
        $+$ & vers la gauche ($x<0$)
    \end{tabular}
\end{center}

% \subsection{Impédance}
% \begin{gather*}
%     Z=\frac{Fk}{\omega} =\frac{F}{c} =\mu c =\sqrt{\mu F}
% \end{gather*}

% \subsection{Puissance}
% \begin{align*}
%     W_{\textit{inst}} &= Z(\omega A)^2\cos^2(\omega t \pm kx +\phi)\\
%     W_{\textit{moy}} &= \frac{Z(\omega A)^2}{2}
% \end{align*}


% \subsection{Interférence dans le plan}
% \begin{center}
%     \includestandalone[scale=1.25]{fig/interference}
% \end{center}
% \[y(P,t)=A_R \cos (\omega t +\phi_R)\]
% \[A_R^2=A_1^2+A_2^2+2A_1 A_2\cos (\psi_2-\psi_1) \]
% \[(\psi_2-\psi_1)=-k(x_2-x_1)+(\phi_2-\phi1)\]

% % \subsubsection{Interférence constructive \hfill$\cos(\psi_2 - \psi_1)>0$}
% % \begin{equation*}
% %     (x_2-x_1)=m\lambda + \frac{\phi_2-\phi_1}{k}
% % \end{equation*}

% % \subsubsection{Interférence destructive \hfill $\cos(\psi_2 - \psi_1)<0$}
% % \begin{equation*}
% %     (x_2-x_1)=(2m+1)\frac{\lambda}{2}+\frac{\phi_2-\phi_1}{k}
% % \end{equation*}


% \begin{tabular}{lll}
%     Type & $\cos(\psi_2 - \psi_1)$ & Équation\\\hline
%     Cons. & $>0$ & \((x_2-x_1)=m\lambda + \frac{\phi_2-\phi_1}{k}\)\\
%     Dest. & $<0$ & \((x_2-x_1)=(2m+1)\frac{\lambda}{2}+\frac{\phi_2-\phi_1}{k}\)
% \end{tabular}



% \subsection{Réflexion et transmission}
% % \subsubsection{Équations}
% \begin{center}
%     \includestandalone[scale=1]{fig/reflex_trans}
% \end{center}
% \begin{tabular}{ll|l}
%     & Milieu 1 & Milieu 2\\\hline
%     incidente & \(y_i=A_i\sin (\omega t - k_1 x)\) &  \multirow{2}{*}{\(y_t=A_t \sin (\omega t-k_2 x) \)} \\
%     réfléchie &\(y_r=A_r\sin (\omega t + k_1 x)\) & \\\hline
%     & \(y_1=y_i+y_r\) & \(y_2=y_t\)
% \end{tabular}

% \subsubsection{Coefficients}
% \begin{tabular}{c|c}
%     Amplitude & Puissance\\\hline
%     \(r=\frac{Z_1-Z_2}{Z_1+Z_2}\) &  \(R=\frac{W_r}{W_i}=\qty(\frac{A_r}{A_i})^2=r^2=\qty(\frac{Z_1-Z_2}{Z_1+Z_2})^2\)\\\rule{-2.5pt}{20pt}
%     \(t=\frac{2Z_1}{Z_1+Z_2}\) & \(T=\frac{W_t}{W_i}=\qty(\frac{A_t}{A_i})^2=\frac{Z_2}{Z_1}t^2=\frac{4Z_1 Z_2}{(Z_1+Z_2)^2}\)
% \end{tabular}
% \begin{tabular}{lll}
%     Caractérisation & Indicateur & Cas extrème\\\hline
%     Réflexion dure & \(Z_2>Z_1\) & \(Z_2=0\)\\
%     Réflexion molle & \(Z_2<Z_1\) & \(Z_2=\alpha\)
% \end{tabular}

% \subsection{Ondes stationnaires}
% % \subsubsection{Équations}
% \begin{align*}
%     y_r &= A\sin (\omega t +kx)\\ 
%     y_i &= -A\sin (\omega t - kx)\\
%     y(x,t)&=2A\sin (kx) \cos (\omega t)
% \end{align*}

% \subsubsection{Noeuds et Ventres}
% \begin{tabular}{l|ll}
%     Noeuds & \(x=\frac{n\lambda}{2}\) & \multirow{2}{*}{\(\mid n = 0,1,2,3,\ldots\)}\\
%     Ventres &  \(x=(2n+1)\frac{\lambda}{4}\) &
% \end{tabular}
% \subsubsection{Combinaison d'ondes stationnaires}
% \begin{tabular}{ll}
%     Fondamantale & \(\lambda_n = \frac{2L}{n}\) \\\rule{-2.5pt}{15pt}
%     Harmoniques & \(f_n=\frac{nc}{2L}\)
% \end{tabular}