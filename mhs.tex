\section{Mouvement harmonique simple}
\vspace{-2\baselineskip}
\begin{center}
    \includestandalone[scale=1.25]{fig/mhs_graph}
\end{center}

% \subsection{Superposition de MHS}
% \subsubsection{Même fréquence et même direction}
% \centering
% \begin{tabular}{l|l}
%     Objectif &  \(x_R(t)=\sum x_i(t)\)\\\hline
%      & \(a_R=\sum A_i\cos(\phi_i)\)\\
%      & \(b_R=\sum A_i\sin(\phi_i)\)\\
%      & \(A_R=\sqrt{a_R^2+b_R^2}\)\\
%      & \(\phi_R=\tan^{-1}\frac{b_R}{a_R}\)\\[8pt]\hline
%     Résultat & \(x_R(t)=A_R\cos(\omega t + \phi_R)\) 
% \end{tabular}

% \raggedright
% \subsubsection{Même fréquence et directions perpendiculaires}
% Solution:
% \begin{equation*}
%     \qty(\frac{x}{A_x})^2+\qty(\frac{y}{A_y})^2-\frac{2xy}{A_x A_y}\cos(\phi_y-\phi_x)=\sin^2(\phi_y-\phi_x)
% \end{equation*}
% Inclinaison:
% \begin{equation*}
%     \tan(2\alpha)=\frac{2A_x A_y}{A_x^2-A_y^2}\cos(\phi_y-\phi_x)
% \end{equation*}

% \subsubsection{Même fréquence et direction quelconques}
% \begin{center}
%     \includestandalone[scale=1]{fig/mhs_rand_dir}
% \end{center}
% \begin{enumerate}[nosep]
%     \item Décomposer le MHS selon X et Y.
%     \item Additionner les composantes de MHS selon X et Y.
%     \item Combiner les résultantes perpendiculaires.
% \end{enumerate}

% \subsubsection{Fréquences différentes et directions perpendiculaires}
% Dépend de: \( \frac{A_x}{A_y}\), \(\frac{\omega_x}{\omega_y}\) et \(\Delta\phi=\phi_y-\phi_x\)\\
% Résolution par analyse de Fourier.